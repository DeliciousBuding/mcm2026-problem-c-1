% Abstract / Summary Sheet
% 鎽樿椤碉紙鍗犵0椤碉紝涓嶈鍏ユ鏂囷級

\begin{abstract}

\emph{Dancing with the Stars} (DWTS) combines judge scores with viewer votes to determine weekly eliminations, yet individual fan vote shares are never directly disclosed. We develop a \textbf{Dual-Core Inversion Engine} to reconstruct latent fan vote distributions from elimination outcomes across all 34 US seasons.

For \emph{percent-rule} seasons (S3--S27), we formulate fan vote inference as a \textbf{Linear Programming (LP)} problem with slack variables, where judge percentages are computed as $J_i / \sum_k J_k$ (sum, not mean). For \emph{rank-rule} seasons (S1--S2, S28--S34), fan vote ranks are treated as \textbf{latent decision variables} and solved via \textbf{Mixed-Integer Linear Programming (MILP)} with permutation constraints. Both solvers incorporate rule changes where they affect feasibility (immunity, double eliminations); for Judges' Save (S28+) we model \emph{bottom-two feasibility} but not the live save criterion, treating undocumented implementation details as modeling uncertainty.

Our framework identifies \textbf{model-data mismatches} in S32 and S33---where constraint slack $S^* > 0$ indicates unexplained outcomes. Possible causes: undocumented rule variations, unmodeled judge-save criteria, or external factors.

Building on inferred votes, \textbf{Cox survival analysis} reveals: (1) professional partner is the dominant survival predictor (likelihood ratio test $\chi^2 = 47.3$, $p < 0.001$; Harrell's C-index improvement $\Delta C = 0.09$); (2) judge scores dominate early rounds while fan momentum drives late-stage outcomes.

We propose a \textbf{Weighted Percent} mechanism ($0.6 \times \text{JudgePct} + 0.4 \times \text{FanPct}$) reducing definite-wrongful eliminations from 40 to \textbf{3} under interval-robust classification. Case studies of Jerry Rice (S2) and Bobby Bones (S27) illustrate key controversies.

\begin{keywords}
inverse problems; linear programming; mixed-integer programming; survival analysis; mechanism design
\end{keywords}

\end{abstract}
