% Section 5: Model 2 - Survival Analysis

\section{Model 2: Survival Analysis}
\label{sec:model2}

We investigate \textbf{what factors predict contestant survival} using Cox Proportional Hazards.

\subsection{Cox Model}

The hazard function:
\begin{equation}
    \lambda(t \mid \mathbf{x}_i) = \lambda_0(t) \exp(\boldsymbol{\beta}^\top \mathbf{x}_i),
\end{equation}
where $\mathrm{HR}_k = \exp(\beta_k)$ is the hazard ratio ($>1$: higher risk; $<1$: protective).

\subsection{Results}

\begin{table}[H]
    \centering
    \caption{Cox Model Results}
    \label{tab:cox_results}
    \begin{threeparttable}
    \begin{tabular}{lccc}
        \toprule
        \textbf{Covariate} & \textbf{HR} & \textbf{95\% CI} & \textbf{$p$} \\
        \midrule
        Athlete (vs.\ Actor) & 0.87 & [0.72, 1.05] & 0.14 \\
        Prior Fame (per unit) & 0.94 & [0.89, 0.99] & 0.02* \\
        Avg Judge (per 5 pts) & 0.62 & [0.55, 0.70] & $<$0.001*** \\
        Avg Fan Vote (per 10\%) & 0.48 & [0.41, 0.56] & $<$0.001*** \\
        Pro Dancer (stratified) & -- & -- & LRT $\chi^2 = 47.3$*** \\
        \bottomrule
    \end{tabular}
    \begin{tablenotes}
        \small \item *$p<0.05$, ***$p<0.001$. LRT: Likelihood Ratio Test comparing stratified vs.\ unstratified Cox model.
    \end{tablenotes}
    \end{threeparttable}
\end{table}

\noindent\textbf{Quantifying Pro Dancer Effect:} We assess professional partner impact via two complementary metrics:
\begin{itemize}[itemsep=0.1em]
    \item \textbf{Likelihood Ratio Test:} $\chi^2 = 2(\mathcal{L}_{\text{stratified}} - \mathcal{L}_{\text{unstratified}}) = 47.3$ with $df = 23$ (number of unique pros), $p < 0.001$.
    \item \textbf{Harrell's C-index:} Model with pro stratification achieves $C = 0.72$; without stratification $C = 0.63$. The improvement $\Delta C = 0.09$ indicates meaningful discriminative gain.
\end{itemize}
These metrics avoid the ambiguity of ``variance explained'' language, which is ill-defined for Cox proportional hazards models.

\noindent\textbf{Key Findings:}
\begin{enumerate}[itemsep=0.1em]
    \item \textbf{Pro Dancer Effect:} Largest factor (LRT $\chi^2 = 47.3$, $\Delta C = 0.09$).
    \item \textbf{Judge Scores:} +5 pts $\to$ 38\% lower hazard (HR=0.62).
    \item \textbf{Fan Votes:} +10\% $\to$ 52\% lower hazard (HR=0.48).
    \item Celebrity type: modest effect ($|\mathrm{HR}-1|<0.2$).
\end{enumerate}

\noindent\textbf{Fan Vote Proxy:} Because fan vote is interval-identified (not point-identified), we use interval midpoints $\hat{v}_i$ as a pragmatic covariate proxy. Sensitivity checks replacing midpoints with lower/upper bounds yield consistent coefficient signs and significance; details in \secref{sec:sensitivity}.

\noindent\textbf{Survival Curve Summary:} Kaplan-Meier analysis by pro dancer tier (top 25\% vs.\ bottom 25\% by historical win rate) shows significant separation ($p < 0.01$, log-rank test). Contestants paired with top-tier pros have median survival of 8 weeks vs.\ 5 weeks for bottom-tier. Full stratified curves omitted due to small per-dancer sample sizes ($n \leq 5$ for most pros); the LRT and C-index metrics above provide more robust quantification.
