% Memo: Model-Based Transparency Protocol

\newpage
\thispagestyle{empty}

\begin{center}
\Large\textbf{MEMORANDUM}
\end{center}

\vspace{0.5em}
\noindent\rule{\textwidth}{1.5pt}
\vspace{0.3em}

\begin{tabular}{@{}ll}
\textbf{TO:}      & Director of Programming, \emph{Dancing with the Stars} \\
\textbf{FROM:}    & Team \#2617892, Data Analytics Consultants \\
\textbf{DATE:}    & January 26, 2026 \\
\textbf{SUBJECT:} & \textbf{Model-Based Transparency Protocol for Voting System Review}
\end{tabular}

\vspace{0.3em}
\noindent\rule{\textwidth}{1.5pt}
\vspace{0.8em}

\subsection*{Executive Summary}

We present a mathematical framework for auditing DWTS elimination outcomes. Our goal is \emph{not} to allege manipulation, but to identify episodes where observed outcomes diverge from model predictions---enabling proactive explanation preparation and system refinement.

\subsection*{Framework Overview}

Our Dual-Core Inversion Engine reconstructs feasible fan vote distributions from elimination data:
\begin{itemize}[itemsep=0.1em]
    \item \textbf{Percent Seasons (S3--S27):} LP with slack variables; judge scores normalized as $J_i / \sum_k J_k$.
    \item \textbf{Rank Seasons (S1--S2, S28--S34):} MILP with permutation constraints on latent fan ranks.
\end{itemize}

\noindent The \textbf{Mismatch Indicator} $S^*$ quantifies the minimum constraint violation needed for feasibility:
\begin{itemize}[itemsep=0.1em]
    \item $S^* = 0$: Model-consistent outcome.
    \item $S^* > 0$: At least one modeling assumption may be violated.
\end{itemize}

\subsection*{Key Findings}

\begin{enumerate}[itemsep=0.1em]
    \item \textbf{S1--S31:} $S^* = 0$ for all seasons (full model consistency).
    \item \textbf{S32--S33:} $S^* > 0$ indicates assumption-data tension. Possible sources:
    \begin{itemize}[itemsep=0.05em]
        \item Judges' Save criteria not captured by rank-sum logic.
        \item Multi-dance week score aggregation differs from assumptions.
        \item Rule transition timing (S28 vs.\ later) is uncertain.
    \end{itemize}
    \item \textbf{Bobby Bones (S27):} Wide feasible interval $[1\%, 91\%]$ illustrates how the 50/50 system can protect low-scoring contestants with dedicated fan bases.
\end{enumerate}

\subsection*{Recommendations}

\begin{enumerate}[itemsep=0.2em]
    \item \textbf{Internal Audit Protocol:}
    \begin{itemize}[itemsep=0.05em]
        \item Run LP/MILP solver before results broadcast (runtime $<5$ sec).
        \item Flag episodes with $S^* > 0.5$ for producer awareness.
        \item Prepare explanatory talking points if media questions arise.
    \end{itemize}
    
    \item \textbf{Weighted Percent Scoring (Optional):}
    \[
        \text{Score}_i = 0.6 \times J_i^\% + 0.4 \times V_i
    \]
    Marketing: ``60\% skill, 40\% popularity.'' Simulations reduce definite-wrongful eliminations from \textbf{40 to 3}.
    
    \item \textbf{Judges' Save Documentation:}
    Publicly announce save criteria (e.g., ``technical improvement'' or ``cumulative performance'') to reduce perceived arbitrariness and improve model fidelity.
\end{enumerate}

\subsection*{Limitations}

\begin{itemize}[itemsep=0.1em]
    \item We do not observe actual vote counts; all estimates are bounds, not exact values.
    \item Mismatches ($S^* > 0$) indicate assumption violations, not manipulation evidence.
    \item Rule details post-S27 are inferred from available sources; official documentation may differ.
\end{itemize}

\vspace{0.5em}
\hfill\textbf{--- Team \#2617892}
