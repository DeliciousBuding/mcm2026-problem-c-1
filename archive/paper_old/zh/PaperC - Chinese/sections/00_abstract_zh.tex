% 摘要 / 概要页

\begin{abstract}

\textbf{受评委与观众反复争议启发,我们将《与星共舞》(DWTS)视为审计问题:} 在不可观测粉丝票的条件下,公开规则是否足以解释淘汰结果?

我们构建\textbf{双核反演引擎}:百分制赛季用 LP 反演可行区间,排名制赛季用 MILP 推断隐含粉丝排名。随后在可行域上采用\textbf{MaxEnt(Hit-and-Run)}并加入高斯随机游走先验,得到后验均值与 HDI,并标记\textbf{Assumption--Data Tension}。

反事实评估给出\textbf{技术公平、观众参与、稳定性与民主赤字}等指标。特征归因采用\textbf{forward-chaining 的 XGBoost + SHAP}并辅以 Cox 生存模型,显示评委表现与舞伴效应是关键驱动因素。

在机制设计上,我们提出\textbf{DAWS 动态自适应权重},并通过帕累托前沿与噪声鲁棒性验证公平性与参与度的权衡。

\begin{keywords}
贝叶斯推断;MaxEnt;Hit-and-Run;线性规划;混合整数规划;XGBoost;SHAP;DAWS
\end{keywords}

\end{abstract}

