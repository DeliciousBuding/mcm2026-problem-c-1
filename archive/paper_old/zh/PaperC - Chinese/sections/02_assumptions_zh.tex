% Section 2: Assumptions (Chinese)

\section{假设}
\label{sec:assumptions}

\subsection{基本假设}
\begin{enumerate}[label=\textbf{H\arabic*.}, leftmargin=2em, itemsep=0.3em]
    \item \textbf{真实投票:} 观众投票反映真实偏好,不进行战略投票。
    \item \textbf{规则一致:} 公布的淘汰结果遵循公开规则;偏离会表现为模型-数据不一致。
    \item \textbf{最小票仓:} 每位选手均至少获得 $\epsilon$ 的票占比(默认 1\%)。
    \item \textbf{规则准确:} 公布的赛制调整(免淘汰、双淘汰、评委拯救)可信。
    \item \textbf{评委独立:} 评委评分不依赖观众投票结果。
    \item \textbf{周次独立:} 观众无法获得真实投票份额并据此策略投票。
\end{enumerate}

\subsection{评估指标定义}
为避免“反演逻辑”与“公平评估”的循环,我们将评估分为两层:

\begin{itemize}[itemsep=0.2em]
    \item \textbf{规则一致性(硬约束)}:在第 2 层反演中,任何有效样本必须复现实测淘汰结果:
    \begin{equation}
        \text{Outcome}_{\text{simulated}} \equiv \text{Outcome}_{\text{observed}}
    \end{equation}
    即模拟中淘汰者必须满足当周规则(最低总分或 bottom-two)。

    \item \textbf{人气偏离(审计指标)}:在规则一致条件下,淘汰者是否并非最低粉丝票:
    \begin{equation}
        \text{IsWrongful} \iff V_{\text{fan, elim}} > \min(\{V_{\text{fan,1}}, \dots, V_{\text{fan,n}}\})
    \end{equation}
    \textit{含义:} 规则一致,但观众意愿被逆转。
\end{itemize}

