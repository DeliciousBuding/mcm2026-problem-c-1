% Abstract / Summary Sheet

\begin{abstract}

\textbf{Motivated by recurring judge--fan disagreements, we frame \emph{Dancing with the Stars} (DWTS) as an audit problem:} can published rules explain eliminations without observing fan vote shares?

We build a \textbf{Dual-Core Inversion Engine}: LP recovers feasible fan-support intervals for percent seasons (S3--S27), while MILP infers latent fan ranks for rank seasons (S1--S2, S28--S34). Over these feasible sets we impose a \textbf{MaxEnt baseline} (Hit-and-Run) and a \textbf{Gaussian random-walk prior}, yielding posterior means and HDIs while flagging \textbf{Assumption--Data Tension}.

Counterfactual simulations quantify \textbf{skill alignment}, \textbf{viewer agency}, \textbf{stability}, and an \textbf{information-theoretic democratic deficit} of rank-only disclosure. For attribution, we use \textbf{forward-chaining} XGBoost with SHAP and a Cox survival check to identify dominant drivers. 

Finally, we propose \textbf{DAWS} (Dynamic Adaptive Weighting System), a transparent rule with smoothness constraints and robustness checks. A Pareto frontier and noise tests show how DAWS balances fairness, engagement, and stability.

\begin{keywords}
Bayesian inference; MaxEnt; Hit-and-Run; linear programming; mixed-integer programming; XGBoost; SHAP; DAWS
\end{keywords}

\end{abstract}
