% Section 2: Assumptions

\section{Assumptions}
\label{sec:assumptions}

\subsection{Assumptions}

\begin{enumerate}[label=\textbf{H\arabic*.}, leftmargin=2em, itemsep=0.3em]
    
    \item \textbf{Sincere Voting.} Viewers vote for favorites, not strategically.
    
    \item \textbf{Truthful Elimination.} Announced outcomes reflect stated rules; deviations are detectable model-data mismatches.
    
    \item \textbf{Vote Floor.} Each contestant receives minimum share $\epsilon > 0$ (default 1\%).
    
    \item \textbf{Rule Accuracy.} Documented production rules (immunity, double elimination, judge save) are correct.
    
    \item \textbf{Judge Independence.} Judges score before votes are tallied.
    
    \item \textbf{Temporal Independence.} Fans cannot condition on hidden vote shares from prior weeks.
    
    \item \textbf{Model-Data Mismatch Detection.} Unmodeled factors (e.g., judge save subjectivity, rule variations) manifest as constraint slack $S^* > 0$.

\end{enumerate}

\subsection{Evaluation Metrics Definition}
To resolve the conflict between ``Inversion Logic'' and ``Fairness Assessment,'' we decouple our metrics into two dimensions:

\begin{itemize}[itemsep=0.2em]
    \item \textbf{Rule Consistency (Feasibility Constraint):} This is the hard constraint for our Layer 2 inversion. We assume the DWTS scoring system operates honestly according to its rules. Therefore, any valid generated sample (parallel universe) must satisfy:
    \begin{equation}
    \text{Outcome}_{\text{simulated}} \equiv \text{Outcome}_{\text{observed}}
    \end{equation}
    This means in our simulation, the eliminated contestant \textit{must} have the lowest total score (or satisfy the specific elimination criteria of that week). Samples violating this are discarded.

    \item \textbf{Popularity Dissonance (Audit Metric):} This is our core measure for ``Wrongful Elimination'' in Layer 3. We investigate whether the eliminated contestant violated the audience's will, \textit{conditional on} the rules being followed.
    \begin{equation}
    \text{IsWrongful} \iff V_{\text{fan, eliminated}} > \min(\{V_{\text{fan, 1}}, ..., V_{\text{fan, n}}\})
    \end{equation}
    \textit{Interpretation:} Although the contestant was eliminated due to the lowest total score (Rule Consistent), they did not actually receive the fewest fan votes (Popularity Dissonance).
\end{itemize}
