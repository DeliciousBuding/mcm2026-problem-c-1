% 备忘录(中文)

\newpage
\thispagestyle{empty}

\begin{center}
\Large\textbf{审计备忘录}
\end{center}

\vspace{0.5em}
\noindent\rule{\textwidth}{1.5pt}
\vspace{0.3em}

\begin{tabular}{@{}ll}
\textbf{收件人:} & 《与星共舞》执行制片人 \\
\textbf{发件人:} & Team \#2617892 --- 建模与审计组 \\
\textbf{日期:} & 2026 年 1 月 30 日 \\
\textbf{主题:} & \textbf{规则审计与动态赛制建议}
\end{tabular}

\vspace{0.3em}
\noindent\rule{\textwidth}{1.5pt}
\vspace{0.8em}

\subsection*{摘要}

我们基于双核反演与贝叶斯后验重建,对 34 季淘汰结果进行审计。核心问题:\textbf{在不可观测粉丝票的情况下,规则是否足以解释淘汰结果?}

\textbf{结论为“是”。} 所有赛季在规则约束下均可解释($S^*=0$)。但规则可解释并不等于确定性——部分周次仍存在较宽的不确定区间。

\subsection*{三项关键发现}

\begin{enumerate}[itemsep=0.2em]
    \item \textbf{规则透明度良好。} \newline
    $S^*$ 全季为 0,说明公开规则足以解释结果。
    
    \item \textbf{结构性不确定性仍在。} \newline
    以 Bobby Bones 为例,后验 HDI 最大宽度约 0.46,表明 50/50 机制存在弱识别区。
    
    \item \textbf{反事实稳定性。} \newline
    Kendall 相关均值约 0.455、逆转率约 0.174。动态权重保持公平信号同时提升观众影响。
\end{enumerate}

\subsection*{建议:动态自适应权重}

\begin{itemize}[itemsep=0.1em]
    \item \textbf{前期(1--5 周):} 评委权重更高(\(\alpha \approx 0.7\))。
    \item \textbf{后期:} 逐步向观众倾斜(\(\alpha \to 0.4\))。
    \item \textbf{静态备选:} 60/40 为稳健基线。
\end{itemize}

\subsection*{范围与限制}

\begin{itemize}[itemsep=0.1em]
    \item 粉丝票不可观测,只能给出后验区间。
    \item 反事实评估假设粉丝行为不随规则改变。
\end{itemize}

\vspace{0.8em}
\hfill\textbf{--- Team \#2617892 建模与审计组}

\vspace{0.3em}
\hfill\textit{详细方法见附录。}

