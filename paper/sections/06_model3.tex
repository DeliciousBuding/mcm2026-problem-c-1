% Section 6: Model 3 - Feature Attribution and Mechanism Design

\section{Model 3: Feature Attribution and Mechanism Design}
\label{sec:model3}

\noindent\textbf{\textit{This section addresses Tasks 3--4:}} We quantify how contestant attributes influence survival with forward-chaining validation, then design a dynamic weighting mechanism guided by a Pareto frontier.

\subsection{Feature Attribution with XGBoost + SHAP}

We model weekly elimination risk using XGBoost (with SHAP explanations) and cross-check with a Cox proportional hazards model. To avoid temporal leakage, we use forward-chaining validation (train on past seasons, test on the next season).

\begin{figure}[H]
    \centering
    \includegraphics[width=0.78\textwidth]{figures/fig_q3_forward_chaining.pdf}
    \caption{\textbf{Predictive scores remain high under forward-chaining.} Predictive scores across seasons without temporal leakage.}
    \label{fig:forward_chaining}
\end{figure}

\begin{figure}[H]
    \centering
    \includegraphics[width=0.85\textwidth]{figures/fig_shap_summary.pdf}
    \caption{\textbf{Judge score statistics dominate the top drivers (Top 15 shown).} Remaining features are deferred to the appendix to reduce visual load.}
    \label{fig:shap_summary}
\end{figure}

\begin{figure}[H]
    \centering
    \includegraphics[width=0.7\textwidth]{figures/fig_shap_interaction.pdf}
    \caption{\textbf{Nonlinear age effects are moderated by pro tier.} This interaction plot is for structural interpretation, not to claim performance gains.}
    \label{fig:shap_interaction}
\end{figure}

\noindent\textbf{Key insight:} Technical performance is the strongest driver, followed by professional partner effects; age contributes nonlinearly with interaction effects.

\subsection{Dynamic Adaptive Weighting}

We propose DAWS (Dynamic Adaptive Weighting System):
\begin{equation}
\alpha_t = \mathrm{clip}\Big(\alpha_0 + \gamma \frac{t}{T} - \eta U_t,\ \alpha_{\min},\alpha_{\max}\Big),
\quad |\alpha_t-\alpha_{t-1}|\le \delta,
\end{equation}
where $U_t$ is an uncertainty proxy from Q1. This enforces transparency and stability.

\begin{figure}[H]
    \centering
    \includegraphics[width=0.7\textwidth]{figures/fig_q4_alpha_schedule.pdf}
    \caption{\textbf{DAWS keeps judge weight within valid bounds.} Uncertainty-adjusted weights with a smoothness constraint.}
    \label{fig:dynamic_schedule}
\end{figure}

\subsection{Pareto Frontier}

We evaluate static weights across $\alpha \in [0.3,0.8]$ and compute two objectives: \textit{judge alignment} (fairness) and \textit{fan influence}. \figref{fig:tradeoffs} shows the trade-off.

\begin{figure}[htbp]
    \centering
    \begin{minipage}[t]{0.48\textwidth}
        \centering
        \includegraphics[width=\textwidth]{figures/fig_q4_pareto_frontier.pdf}
    \end{minipage}
    \hfill
    \begin{minipage}[t]{0.48\textwidth}
        \centering
        \includegraphics[width=\textwidth]{figures/fig_ternary_tradeoff.pdf}
    \end{minipage}
    \caption{\textbf{DAWS sits in the balanced region of the design space.} Left: Pareto frontier for judge alignment vs.\ fan influence. Right: Ternary view of fairness, agency, and stability; DAWS ($\star$) targets the balanced region.}
    \label{fig:tradeoffs}
\end{figure}
\begin{figure}[H]
    \centering
    \includegraphics[width=0.75\textwidth]{figures/fig_q4_noise_robustness.pdf}
    \caption{\textbf{DAWS remains stable under small noise.} Elimination flips under score/share perturbations.}
    \label{fig:daws_robust}
\end{figure}

\noindent\textbf{Recommendation:} Use DAWS to preserve early technical screening while maintaining late-stage audience ownership and robustness.
