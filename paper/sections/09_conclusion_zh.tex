% 结论(中文)

\section{结论}
\label{sec:conclusion_zh}

\subsection{主要结论}
\begin{enumerate}[itemsep=0.2em]
    \item \textbf{双核反演 + MaxEnt/Bayesian:} LP/MILP 提供可行区间,Hit-and-Run + 随机游走先验输出后验均值与 HDI。
    \item \textbf{Assumption--Data Tension:} $S^*$ 与可行质量标记弱识别周次。
    \item \textbf{反事实评估:} 技术公平、观众参与、稳定性与民主赤字量化赛制差异。
    \item \textbf{特征归因:} forward-chaining 的 XGBoost + SHAP(Cox 互证)揭示关键驱动因素。
    \item \textbf{机制设计:} DAWS 在公平、参与与稳定性之间取得平衡。
\end{enumerate}

\subsection{建议}
\begin{enumerate}[itemsep=0.2em]
    \item \textbf{赛前审计:} 赛前运行反演与后验分析,定位高不确定周次。
    \item \textbf{DAWS 权重:} 不确定性调节的动态权重并保持平滑。
    \item \textbf{评委拯救透明化:} 公布判断准则以降低争议。
\end{enumerate}

\subsection{局限性}
\begin{itemize}[itemsep=0.1em]
    \item 未建模外部舆情冲击。
    \item 排名制赛季仍存在弱识别问题。
    \item 反事实假设粉丝行为不随规则变化。
\end{itemize}

\subsection{未来工作}
引入外部人气先验、自适应 MCMC,并构建实时审计面板。

\vspace{0.5em}
\noindent\textbf{结论要点:} 规则一致的贝叶斯重建可提供可解释、公平的审计结果,并支持动态赛制设计。

