% Section 5: Model 2 (Chinese)

\section{模型二:反事实赛制评估}
\label{sec:model2_zh}

\noindent\textbf{\textit{对应问题 2:}} 固定粉丝意愿与评委分数,替换规则权重,评估赛制公平性与稳定性。

\subsection{评估指标}
\begin{itemize}[itemsep=0.2em]
    \item \textbf{技术公平}:机制分数与评委分数的 Kendall 相关。
    \item \textbf{观众参与}:是否与粉丝最低票一致。
    \item \textbf{稳定性}:淘汰分布的熵稳定度。
    \item \textbf{民主赤字}:排名制相对百分制的反转概率。
\end{itemize}

\subsection{结果}
\begin{figure}[H]
    \centering
    \includegraphics[width=0.72\textwidth]{figures/fig_q2_counterfactual_matrix.pdf}
    \caption{\textbf{DAWS 基础规则在多指标下最接近平衡(高亮)。}}
    \label{fig:cf_matrix_zh}
\end{figure}

\begin{figure}[H]
    \centering
    \includegraphics[width=0.68\textwidth]{figures/fig_q2_save_sensitivity.pdf}
    \caption{\textbf{民主赤字随 Judge Save 软度变化。} 阴影带为 S28+ 参考区间。}
    \label{fig:save_sensitivity_zh}
\end{figure}

\begin{figure}[H]
    \centering
    \includegraphics[width=0.72\textwidth]{figures/fig_q2_democratic_deficit.pdf}
    \caption{\textbf{S28 规则切换后赤字走势被重新分段(阴影)。} 排名制的信息损失。}
    \label{fig:deficit_zh}
\end{figure}
