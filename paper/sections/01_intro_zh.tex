% 引言(中文)

\section{引言}
\label{sec:intro_zh}

\subsection{背景}

\textbf{受评委与观众争议启发},我们将《与星共舞》视为审计问题:在不可观测粉丝票的条件下,公开规则是否足以解释淘汰结果?

节目规则在 34 季中多次变化(见图 \ref{fig:show_process}),因此需要在不同信息结构下进行反演。

\begin{figure}[h]
    \centering
    \includegraphics[width=0.95\textwidth]{figures/fig_dwts_show_process.pdf}
    \vspace{-1.0em}
    \caption{\textbf{审计将评委、粉丝与淘汰结果贯通于不同规则时代。} 每周选手表演后接受评委和粉丝打分,综合排名末两位进入淘汰区,评委可救一人。}
    \vspace{-0.8em}
    \label{fig:show_process}
\end{figure}

\begin{figure}[h]
    \centering
    \includegraphics[width=0.95\textwidth]{figures/fig_sankey_audit.pdf}
    \vspace{-0.8em}
    \caption{\textbf{结果模式随评委–粉丝信号对齐而变化。} 选手在评委/粉丝信号下流向“安全/拯救/淘汰”。}
    \vspace{-0.8em}
    \label{fig:sankey_audit}
\end{figure}

\subsection{问题描述}
\begin{enumerate}[label=\arabic*., nosep, leftmargin=*]
    \item \textbf{反演粉丝票(任务1--2):} 基于淘汰结果推断可行区间。
    \item \textbf{规则变动影响(任务2):} 免疫、双淘汰、评委拯救的影响。
    \item \textbf{成功因素(任务3):} 何种属性影响生存周数。
    \item \textbf{机制评估(任务4):} 现有赛制是否公平。
    \item \textbf{机制设计(任务5):} 如何优化公平性与参与度。
\end{enumerate}

\subsection{贡献}
\begin{enumerate}[itemsep=0.1em]
    \item \textbf{双核反演 + MaxEnt/Bayesian:} LP/MILP 反演区间,Hit-and-Run + 随机游走先验输出均值与 HDI。
    \item \textbf{Assumption--Data Tension:} 以 $S^*$ 与可行质量衡量识别强度。
    \item \textbf{反事实评估:} 技术公平、观众参与、稳定性与民主赤字。
    \item \textbf{特征归因:} forward-chaining 的 XGBoost + SHAP(Cox 互证)。
    \item \textbf{DAWS 机制设计:} 不确定性调节的动态权重与鲁棒性验证。
\end{enumerate}
