% AI工具使用说明 (COMAP 2026)

\section*{AI工具报告}

\addcontentsline{toc}{section}{AI工具报告}

\label{appendix:ai_zh}



\textbf{使用的工具:} Claude Opus 4.5(代码架构、算法实现)、ChatGPT-4o(语法)、GitHub Copilot(代码辅助)、DeepL(翻译)。



\textbf{使用摘要:}

\begin{table}[H]

\centering\small

\begin{tabularx}{\textwidth}{l l X}

\toprule

\textbf{工具} & \textbf{任务} & \textbf{用法} \\

\midrule

Claude & 模型设计 & LP/MILP 框架公式化 \\
Claude & 代码实现 & Python \texttt{dwts\_model} 软件包 \\

Claude & 数据分析 & Money Plot 不匹配检测 \\

ChatGPT & 文本编写 & 语法润色(技术内容由团队完成) \\

\bottomrule

\end{tabularx}

\end{table}



\textbf{验证工作:} 所有的AI输出都经过了审查和测试。关键结果(确定性错误淘汰从 40 减少到 3,S32/S33 不匹配)得到了独立验证。LP/MILP 公式均对照优化教科书进行了检查。



\textbf{纯人工贡献:} 问题定义、数据收集、结果解构、政策建议、最终参数选择 ($\alpha = 0.6$, $\epsilon = 1\%$)。



\textbf{声明:} AI工具有效加速了工作流程;所有实质性的智力贡献均反映了人类团队的判断。



\vspace{1em}

\noindent\textbf{签名:} Team \#2617892

