% Section 4: Model 1 (Chinese)

\section{模型一:双核反演与后验重建}
\label{sec:model1_zh}

\noindent\textbf{\textit{对应问题 1--2:}} 我们在公开规则下反演粉丝票可行区间,并通过 MaxEnt + 随机游走先验重建后验分布。

\subsection{问题形式}
设第 $s$ 赛季第 $t$ 周活跃集合为 $\mathcal{C}_{s,t}$,观测评委分与淘汰结果 $E$。粉丝票份额满足:
\begin{equation}
    \sum_{i=1}^{n} v_i = 1, \quad v_i \ge \epsilon.
\end{equation}

\subsection{双核反演}

\begin{figure}[H]
    \centering
    \includegraphics[width=0.9\textwidth]{figures/fig_dwts_flowchart_vector.pdf}
    \caption{\textbf{不确定性在双核引擎中向下游传播。} 百分制赛季使用LP核心,排名制赛季使用MILP核心。}
    \label{fig:dual_engine_zh}
\end{figure}

- 百分制赛季:线性规划(LP)输出区间 $[L_i,U_i]$。  
- 排名制赛季:混合整数规划(MILP)推断隐含排名并采用 bottom-two 约束(Judge Save)。

\subsection{Assumption--Data Tension}
通过最小松弛 $S^*$ 与可行质量(acceptance rate)识别假设与数据的张力,而非指控操纵。

\begin{figure}[H]
    \centering
    \includegraphics[width=0.86\textwidth]{figures/fig_q1_uncertainty_heatmap.pdf}
    \caption{\textbf{弱识别点集中在少数周次(白圈标注)。} 反映可行质量与识别强度;这些点用于后续案例分析。}
    \label{fig:money_plot_zh}
\end{figure}

\subsection{MaxEnt + 随机游走先验}
在可行域上采用 Hit-and-Run 的 MaxEnt 采样,并加入时间平滑先验:
\begin{equation}
    p(\mathbf{v}_t \mid \mathbf{v}_{t-1}) \propto \exp\left(-\frac{\lVert \mathbf{v}_t-\mathbf{v}_{t-1}\rVert^2}{2\sigma^2}\right).
\end{equation}
通过重要性重采样与 warm start 输出后验均值与 95\% HDI。

\begin{figure}[H]
    \centering
    \includegraphics[width=0.86\textwidth]{figures/fig_q1_hdi_band.pdf}
    \caption{\textbf{淘汰并不总是对应最低粉丝支持。} 红色 $\times$ 标注淘汰周,线条加粗表示被淘汰选手。}
    \label{fig:hdi_band_zh}
\end{figure}


\subsection{小结}
\begin{itemize}[itemsep=0.2em]
    \item \textbf{输出:} 可行区间、后验均值与 95\% HDI。
    \item \textbf{识别强度:} 接受率与边缘概率用于量化不确定性。
    \item \textbf{下游:} 后验样本进入反事实评估与机制设计。
\end{itemize}

