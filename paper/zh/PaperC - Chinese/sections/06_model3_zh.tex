% Section 6: Model 3 (Chinese)

\section{模型三:特征归因与赛制设计}
\label{sec:model3_zh}

\noindent\textbf{\textit{对应问题 3--4:}} 使用 forward-chaining 的 XGBoost+SHAP 解释选手生存因素,并设计 DAWS 动态加权赛制。

\subsection{特征归因(XGBoost + SHAP)}

\begin{figure}[H]
    \centering
    \includegraphics[width=0.78\textwidth]{figures/fig_q3_forward_chaining.pdf}
    \caption{\textbf{前向验证下预测分数依然保持较高水平。} 逐赛季 forward-chaining 评估。}
    \label{fig:forward_chaining_zh}
\end{figure}

\begin{figure}[H]
    \centering
    \includegraphics[width=0.85\textwidth]{figures/fig_shap_summary.pdf}
    \caption{\textbf{评委分统计是最主要驱动(主文仅保留 Top 15)。} 其余特征移至附录以降低视觉负担。}
    \label{fig:shap_summary_zh}
\end{figure}

\begin{figure}[H]
    \centering
    \includegraphics[width=0.7\textwidth]{figures/fig_shap_interaction.pdf}
    \caption{\textbf{年龄效应呈非线性,并受舞伴层级调节。} 该图用于结构性解释,而非提升预测性能。}
    \label{fig:shap_interaction_zh}
\end{figure}

\begin{figure}[H]
    \centering
    \includegraphics[width=0.72\textwidth]{figures/fig_shap_waterfall.pdf}
    \caption{\textbf{SHAP 瀑布图(案例)。} 单周淘汰风险的关键驱动分解。}
    \label{fig:shap_waterfall_zh}
\end{figure}

\subsection{DAWS 动态权重机制}

\begin{equation}
\alpha_t = \mathrm{clip}\Big(\alpha_0 + \gamma \frac{t}{T} - \eta U_t,\ \alpha_{\min},\alpha_{\max}\Big), \quad
|\alpha_t-\alpha_{t-1}|\le \delta,
\end{equation}
其中 $U_t$ 为不确定性代理。

\begin{figure}[H]
    \centering
    \includegraphics[width=0.7\textwidth]{figures/fig_q4_alpha_schedule.pdf}
    \caption{\textbf{DAWS 将评委权重稳定在合法区间内。} 受不确定性调节并满足平滑约束。}
    \label{fig:dynamic_schedule_zh}
\end{figure}

\subsection{帕累托前沿}

\begin{figure}[htbp]
    \centering
    \begin{minipage}[t]{0.48\textwidth}
        \centering
        \includegraphics[width=\textwidth]{figures/fig_q4_pareto_frontier.pdf}
    \end{minipage}
    \hfill
    \begin{minipage}[t]{0.48\textwidth}
        \centering
        \includegraphics[width=\textwidth]{figures/fig_ternary_tradeoff.pdf}
    \end{minipage}
    \caption{\textbf{DAWS 位于设计空间的平衡区。} 左:评委一致性 vs.\ 粉丝影响力的帕累托前沿;右:公平、参与与稳定性的三元视图。}
    \label{fig:tradeoffs_zh}
\end{figure}
\begin{figure}[H]
    \centering
    \includegraphics[width=0.75\textwidth]{figures/fig_q4_noise_robustness.pdf}
    \caption{\textbf{DAWS 在小幅扰动下保持稳定。} 权重机制对扰动的稳定性。}
    \label{fig:daws_robust_zh}
\end{figure}

\noindent\textbf{建议:} 采用 DAWS 以兼顾公平、参与与稳定性。
