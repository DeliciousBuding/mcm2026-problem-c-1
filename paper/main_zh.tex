% MCM/ICM 2026 Chinese paper main file
% !TEX program = xelatex
% !BIB program = biber
\documentclass{mcmthesis}

% Hyperref chapter counter compatibility
\newcounter{chapter}
\renewcommand{\thechapter}{\arabic{chapter}}

\mcmsetup{
    tcn = {2617892},
    problem = {C},
    sheet = true,
    titleinsheet = true,
    keywordsinsheet = true,
    titlepage = false
}

% TOC toggle
\makeatletter
\@ifundefined{showtoctrue}{}{\showtoctrue}
\makeatother

% === Packages ===
\usepackage{xeCJK}
% 如果目标字体缺少粗体/斜体,使用 xeCJK 的自动伪粗/伪斜设置
\xeCJKsetup{AutoFakeBold=true, AutoFakeSlant=true}
\usepackage{fontspec}
\defaultfontfeatures{Ligatures=TeX,Scale=MatchLowercase}
\setmainfont{TeX Gyre Pagella}
\setsansfont{TeX Gyre Heros}
\setmonofont{Latin Modern Mono}
% Robust CJK font fallbacks to avoid missing-font build errors.
% Prefer Noto CJK families when available (better bold/italic support)
\IfFontExistsTF{Noto Serif CJK SC}{\setCJKmainfont{Noto Serif CJK SC}}{%
  \IfFontExistsTF{SimSun}{\setCJKmainfont{SimSun}}{%
    \IfFontExistsTF{Songti SC}{\setCJKmainfont{Songti SC}}{%
      \IfFontExistsTF{STSong}{\setCJKmainfont{STSong}}{%
        \setCJKmainfont{FandolSong-Regular}}}}}
\IfFontExistsTF{Noto Sans CJK SC}{\setCJKsansfont{Noto Sans CJK SC}}{%
  \IfFontExistsTF{SimHei}{\setCJKsansfont{SimHei}}{%
    \IfFontExistsTF{Microsoft YaHei}{\setCJKsansfont{Microsoft YaHei}}{%
      \IfFontExistsTF{PingFang SC}{\setCJKsansfont{PingFang SC}}{%
        \setCJKsansfont{FandolHei-Regular}}}}}
\IfFontExistsTF{FangSong}{\setCJKmonofont{FangSong}}{%
  \IfFontExistsTF{SimSun}{\setCJKmonofont{SimSun}}{%
    \IfFontExistsTF{Noto Sans Mono CJK SC}{\setCJKmonofont{Noto Sans Mono CJK SC}}{%
      \setCJKmonofont{FandolFang-Regular}}}}

\usepackage{microtype}
\setlength{\emergencystretch}{2em}
\usepackage{zref-abspage}
\usepackage{float}
\usepackage{geometry}
\setlength{\headheight}{15pt}
\usepackage{amsmath, amssymb, amsthm}
\usepackage{mathtools}
\usepackage{booktabs}
\usepackage{tabularx}
\usepackage{longtable}
\usepackage{multirow}
\usepackage{array}
\usepackage{threeparttable}
\usepackage[ruled, vlined, linesnumbered]{algorithm2e}
\SetAlgorithmName{Algorithm}{algorithm}{List of Algorithms}
\SetKwInput{KwIn}{Input}
\SetKwInput{KwOut}{Output}
\SetKwInput{KwData}{Data}
\SetKwInput{KwResult}{Result}
\usepackage{listings}
\lstset{
    basicstyle=\small\ttfamily,
    keywordstyle=\color{blue},
    commentstyle=\color{gray},
    numbers=left,
    numberstyle=\tiny\color{gray},
    frame=single,
    breaklines=true,
    tabsize=4
}
\usepackage{graphicx}
\usepackage{subcaption}
\usepackage{xcolor}
\usepackage{tikz}
\usetikzlibrary{shapes, arrows, positioning, calc}

% =============================================================================
% DWTS Paper Palette - 统一配色方案
% =============================================================================
\definecolor{dwts-proposed}{HTML}{2C4E5F}   % 青蓝 - 新机制/推荐
\definecolor{dwts-baseline}{HTML}{0B2A3A}   % 藏蓝 - 基准/原机制

\definecolor{dwts-warning}{HTML}{D35A3A}    % 深蓝 - 警示/不匹配

\definecolor{dwts-warning2}{HTML}{A9B2BB}   % 灰蓝 - 次级警示
\definecolor{dwts-fill}{HTML}{9CB8C8}       % 浅蓝 - 填充/区间
\definecolor{dwts-accent}{HTML}{D35A3A}     % ?? - ????
\definecolor{dwts-aux}{HTML}{2C4E5F}        % ?? - ??

\usepackage{enumitem}
\setlist[enumerate]{itemsep=0pt, parsep=0pt}

% 彩色框(用于重点内容)- 使用统一配色
\usepackage{tcolorbox}
\tcbuselibrary{skins, breakable}

% 预定义 tcolorbox 样式
\newtcolorbox{proposedbox}[1][]{
    colback=dwts-proposed!10!white,
    colframe=dwts-proposed!80!black,
    fonttitle=\bfseries,
    #1
}
\newtcolorbox{warningbox}[1][]{
    colback=dwts-warning!10!white,
    colframe=dwts-warning!80!black,
    fonttitle=\bfseries,
    #1
}
\newtcolorbox{baselinebox}[1][]{
    colback=dwts-baseline!5!white,
    colframe=dwts-baseline!70!black,
    fonttitle=\bfseries,
    #1
}
\newtcolorbox{insightbox}[1][]{
    colback=dwts-fill!30!white,
    colframe=dwts-aux!80!black,
    fonttitle=\bfseries,
    #1
}

% 超链接(最后加载)- 使用统一配色
\usepackage{hyperref}
\hypersetup{
    colorlinks=true,
    linkcolor=dwts-baseline,
    citecolor=dwts-aux,
    urlcolor=dwts-proposed
}
\providecommand{\theHpage}{\arabic{page}}
\usepackage{tocloft}
\setlength{\cftbeforesecskip}{2pt}
\setlength{\cftbeforesubsecskip}{1pt}
\usepackage[backend=biber, style=ieee, sorting=none]{biblatex}
\addbibresource{ref.bib}

% === Custom commands ===
\newcounter{mainpages}
\makeatletter
\newcommand{\savemainpages}{%
    \immediate\write\@auxout{\string\setcounter{mainpages}{\the\value{page}}}%
}
\makeatother
\newcommand{\figref}[1]{Fig.~\ref{#1}}
\newcommand{\tabref}[1]{Table~\ref{#1}}
\newcommand{\eqnref}[1]{Eq.~(\ref{#1})}
\newcommand{\algref}[1]{Algorithm~\ref{#1}}
\newcommand{\secref}[1]{Section~\ref{#1}}

% === Meta ===
\title{基于淘汰数据的观众投票反演:\\
具备故障监测能力的双核反演框架\\
及针对《与星共舞》的机制设计}
\author{Team \#2617892}

\begin{document}
% Summary Sheet
\pagenumbering{arabic}
\setcounter{page}{0}
\pagestyle{empty}
% 摘要 / 概要页

\begin{abstract}

\textbf{受评委与观众反复争议启发,我们将《与星共舞》(DWTS)视为审计问题:} 在不可观测粉丝票的条件下,公开规则是否足以解释淘汰结果?

我们构建\textbf{双核反演引擎}:百分制赛季用 LP 反演可行区间,排名制赛季用 MILP 推断隐含粉丝排名。随后在可行域上采用\textbf{MaxEnt(Hit-and-Run)}并加入高斯随机游走先验,得到后验均值与 HDI,并标记\textbf{Assumption--Data Tension}。

反事实评估给出\textbf{技术公平、观众参与、稳定性与民主赤字}等指标。特征归因采用\textbf{forward-chaining 的 XGBoost + SHAP}并辅以 Cox 生存模型,显示评委表现与舞伴效应是关键驱动因素。

在机制设计上,我们提出\textbf{DAWS 动态自适应权重},并通过帕累托前沿与噪声鲁棒性验证公平性与参与度的权衡。

\begin{keywords}
贝叶斯推断;MaxEnt;Hit-and-Run;线性规划;混合整数规划;XGBoost;SHAP;DAWS
\end{keywords}

\end{abstract}


\maketitle
\thispagestyle{empty}

% Memo
\newpage
\setcounter{page}{1}
\pagestyle{fancy}
\fancyhf{}
\fancyhead[L]{\small Team \#2617892}
\fancyhead[R]{\small 第\thepage 页,共\themainpages 页}
\fancyfoot[C]{}
% 备忘录(中文)

\newpage
\thispagestyle{empty}

\begin{center}
\Large\textbf{审计备忘录}
\end{center}

\vspace{0.5em}
\noindent\rule{\textwidth}{1.5pt}
\vspace{0.3em}

\begin{tabular}{@{}ll}
\textbf{收件人:} & 《与星共舞》执行制片人 \\
\textbf{发件人:} & Team \#2617892 --- 建模与审计组 \\
\textbf{日期:} & 2026 年 1 月 30 日 \\
\textbf{主题:} & \textbf{规则审计与动态赛制建议}
\end{tabular}

\vspace{0.3em}
\noindent\rule{\textwidth}{1.5pt}
\vspace{0.8em}

\subsection*{摘要}

我们基于双核反演与贝叶斯后验重建,对 34 季淘汰结果进行审计。核心问题:\textbf{在不可观测粉丝票的情况下,规则是否足以解释淘汰结果?}

\textbf{结论为“是”。} 所有赛季在规则约束下均可解释($S^*=0$)。但规则可解释并不等于确定性——部分周次仍存在较宽的不确定区间。

\subsection*{三项关键发现}

\begin{enumerate}[itemsep=0.2em]
    \item \textbf{规则透明度良好。} \newline
    $S^*$ 全季为 0,说明公开规则足以解释结果。
    
    \item \textbf{结构性不确定性仍在。} \newline
    以 Bobby Bones 为例,后验 HDI 最大宽度约 0.46,表明 50/50 机制存在弱识别区。
    
    \item \textbf{反事实稳定性。} \newline
    Kendall 相关均值约 0.455、逆转率约 0.174。动态权重保持公平信号同时提升观众影响。
\end{enumerate}

\subsection*{建议:动态自适应权重}

\begin{itemize}[itemsep=0.1em]
    \item \textbf{前期(1--5 周):} 评委权重更高(\(\alpha \approx 0.7\))。
    \item \textbf{后期:} 逐步向观众倾斜(\(\alpha \to 0.4\))。
    \item \textbf{静态备选:} 60/40 为稳健基线。
\end{itemize}

\subsection*{范围与限制}

\begin{itemize}[itemsep=0.1em]
    \item 粉丝票不可观测,只能给出后验区间。
    \item 反事实评估假设粉丝行为不随规则改变。
\end{itemize}

\vspace{0.8em}
\hfill\textbf{--- Team \#2617892 建模与审计组}

\vspace{0.3em}
\hfill\textit{详细方法见附录。}



% Contents
\newpage
\renewcommand{\contentsname}{目录}
\tableofcontents

% Main
\newpage
\pagestyle{fancy}
\fancyhf{}
\fancyhead[L]{\small Team \#2617892}
\fancyhead[R]{\small 第\thepage 页,共\themainpages 页}
\fancyfoot[C]{}
% 引言(中文)

\section{引言}
\label{sec:intro_zh}

\subsection{背景}

\textbf{受评委与观众争议启发},我们将《与星共舞》视为审计问题:在不可观测粉丝票的条件下,公开规则是否足以解释淘汰结果?

节目规则在 34 季中多次变化(见图 \ref{fig:show_process}),因此需要在不同信息结构下进行反演。

\begin{figure}[h]
    \centering
    \includegraphics[width=0.95\textwidth]{figures/fig_dwts_show_process.pdf}
    \vspace{-1.0em}
    \caption{\textbf{审计将评委、粉丝与淘汰结果贯通于不同规则时代。} 每周选手表演后接受评委和粉丝打分,综合排名末两位进入淘汰区,评委可救一人。}
    \vspace{-0.8em}
    \label{fig:show_process}
\end{figure}

\begin{figure}[h]
    \centering
    \includegraphics[width=0.95\textwidth]{figures/fig_sankey_audit.pdf}
    \vspace{-0.8em}
    \caption{\textbf{结果模式随评委–粉丝信号对齐而变化。} 选手在评委/粉丝信号下流向“安全/拯救/淘汰”。}
    \vspace{-0.8em}
    \label{fig:sankey_audit}
\end{figure}

\subsection{问题描述}
\begin{enumerate}[label=\arabic*., nosep, leftmargin=*]
    \item \textbf{反演粉丝票(任务1--2):} 基于淘汰结果推断可行区间。
    \item \textbf{规则变动影响(任务2):} 免疫、双淘汰、评委拯救的影响。
    \item \textbf{成功因素(任务3):} 何种属性影响生存周数。
    \item \textbf{机制评估(任务4):} 现有赛制是否公平。
    \item \textbf{机制设计(任务5):} 如何优化公平性与参与度。
\end{enumerate}

\subsection{贡献}
\begin{enumerate}[itemsep=0.1em]
    \item \textbf{双核反演 + MaxEnt/Bayesian:} LP/MILP 反演区间,Hit-and-Run + 随机游走先验输出均值与 HDI。
    \item \textbf{Assumption--Data Tension:} 以 $S^*$ 与可行质量衡量识别强度。
    \item \textbf{反事实评估:} 技术公平、观众参与、稳定性与民主赤字。
    \item \textbf{特征归因:} forward-chaining 的 XGBoost + SHAP(Cox 互证)。
    \item \textbf{DAWS 机制设计:} 不确定性调节的动态权重与鲁棒性验证。
\end{enumerate}

% Section 2: Assumptions (Chinese)

\section{假设}
\label{sec:assumptions}

\subsection{基本假设}
\begin{enumerate}[label=\textbf{H\arabic*.}, leftmargin=2em, itemsep=0.3em]
    \item \textbf{真实投票:} 观众投票反映真实偏好,不进行战略投票。
    \item \textbf{规则一致:} 公布的淘汰结果遵循公开规则;偏离会表现为模型-数据不一致。
    \item \textbf{最小票仓:} 每位选手均至少获得 $\epsilon$ 的票占比(默认 1\%)。
    \item \textbf{规则准确:} 公布的赛制调整(免淘汰、双淘汰、评委拯救)可信。
    \item \textbf{评委独立:} 评委评分不依赖观众投票结果。
    \item \textbf{周次独立:} 观众无法获得真实投票份额并据此策略投票。
\end{enumerate}

\subsection{评估指标定义}
为避免“反演逻辑”与“公平评估”的循环,我们将评估分为两层:

\begin{itemize}[itemsep=0.2em]
    \item \textbf{规则一致性(硬约束)}:在第 2 层反演中,任何有效样本必须复现实测淘汰结果:
    \begin{equation}
        \text{Outcome}_{\text{simulated}} \equiv \text{Outcome}_{\text{observed}}
    \end{equation}
    即模拟中淘汰者必须满足当周规则(最低总分或 bottom-two)。

    \item \textbf{人气偏离(审计指标)}:在规则一致条件下,淘汰者是否并非最低粉丝票:
    \begin{equation}
        \text{IsWrongful} \iff V_{\text{fan, elim}} > \min(\{V_{\text{fan,1}}, \dots, V_{\text{fan,n}}\})
    \end{equation}
    \textit{含义:} 规则一致,但观众意愿被逆转。
\end{itemize}


% 符号说明(中文)

\section{符号说明}
\label{sec:notations_zh}

\begin{table}[H]
    \centering
    \caption{主要符号说明}
    \label{tab:notations_zh}
    \begin{tabularx}{\textwidth}{c X c}
        \toprule
        \textbf{符号} & \textbf{含义} & \textbf{范围} \\
        \midrule
        $s$ & 赛季编号 & $s \in \{1,\ldots,34\}$ \\
        $t$ & 周次编号 & $t \in \{1,\ldots,T_s\}$ \\
        $i$ & 选手编号 & -- \\
        $\mathcal{C}_{s,t}$ & 第 $s$ 赛季第 $t$ 周活跃集合 & -- \\
        $n_{s,t}$ & 活跃选手数量 & -- \\
        \midrule
        $J_{i,t}$ & 评委分数总和 & [0, 40] \\
        $Jpct_{i,t}$ & 评委百分比 $J_{i,t}/\sum_k J_{k,t}$ & $[0,1]$ \\
        $E_t$ & 当周淘汰选手 & -- \\
        $r^{\mathrm{fan}}_{i,t}$ & 粉丝排名(排名制隐变量) & $\{1,\ldots,n_{s,t}\}$ \\
        \midrule
        $v_{i,t}$ & 粉丝票份额 & $[0,1]$, $\sum_i v_{i,t}=1$ \\
        $\hat{v}_{i,t}$ & 后验均值 & $[0,1]$ \\
        $[v^{\min}_{i,t}, v^{\max}_{i,t}]$ & 可行区间 & $[0,1]$ \\
        $\mathrm{HDI}_{i,t}$ & 95\% HDI 区间 & $[0,1]$ \\
        \midrule
        $\epsilon$ & 最小份额约束 & $[0,0.1]$ \\
        $S^*$ & 最小松弛(Assumption--Data Tension) & $\ge 0$ \\
        $\sigma$ & 随机游走平滑系数 & $>0$ \\
        $\alpha$ & 评委权重 & $[0,1]$ \\
        \midrule
        $T_i$ & 选手存活周数 & weeks \\
        $f(\mathbf{x}_i)$ & XGBoost 预测值 & weeks \\
        $\phi_j$ & 特征 $j$ 的 SHAP 值 & -- \\
        \midrule
        $Jpct_{i,t}$ & 评委百分比 & $[0,1]$ \\
        $S_{i,t}^{\mathrm{wp}}$ & 加权得分 $\alpha Jpct_{i,t} + (1-\alpha)v_{i,t}$ & $[0,1]$ \\
        \bottomrule
    \end{tabularx}
\end{table}


% Section 4: Model 1 (Chinese)

\section{模型一:双核反演与后验重建}
\label{sec:model1_zh}

\noindent\textbf{\textit{对应问题 1--2:}} 我们在公开规则下反演粉丝票可行区间,并通过 MaxEnt + 随机游走先验重建后验分布。

\subsection{问题形式}
设第 $s$ 赛季第 $t$ 周活跃集合为 $\mathcal{C}_{s,t}$,观测评委分与淘汰结果 $E$。粉丝票份额满足:
\begin{equation}
    \sum_{i=1}^{n} v_i = 1, \quad v_i \ge \epsilon.
\end{equation}

\subsection{双核反演}

\begin{figure}[H]
    \centering
    \includegraphics[width=0.9\textwidth]{figures/fig_dwts_flowchart_vector.pdf}
    \caption{\textbf{不确定性在双核引擎中向下游传播。} 百分制赛季使用LP核心,排名制赛季使用MILP核心。}
    \label{fig:dual_engine_zh}
\end{figure}

- 百分制赛季:线性规划(LP)输出区间 $[L_i,U_i]$。  
- 排名制赛季:混合整数规划(MILP)推断隐含排名并采用 bottom-two 约束(Judge Save)。

\subsection{Assumption--Data Tension}
通过最小松弛 $S^*$ 与可行质量(acceptance rate)识别假设与数据的张力,而非指控操纵。

\begin{figure}[H]
    \centering
    \includegraphics[width=0.86\textwidth]{figures/fig_q1_uncertainty_heatmap.pdf}
    \caption{\textbf{弱识别点集中在少数周次(白圈标注)。} 反映可行质量与识别强度;这些点用于后续案例分析。}
    \label{fig:money_plot_zh}
\end{figure}

\subsection{MaxEnt + 随机游走先验}
在可行域上采用 Hit-and-Run 的 MaxEnt 采样,并加入时间平滑先验:
\begin{equation}
    p(\mathbf{v}_t \mid \mathbf{v}_{t-1}) \propto \exp\left(-\frac{\lVert \mathbf{v}_t-\mathbf{v}_{t-1}\rVert^2}{2\sigma^2}\right).
\end{equation}
通过重要性重采样与 warm start 输出后验均值与 95\% HDI。

\begin{figure}[H]
    \centering
    \includegraphics[width=0.86\textwidth]{figures/fig_q1_hdi_band.pdf}
    \caption{\textbf{淘汰并不总是对应最低粉丝支持。} 红色 $\times$ 标注淘汰周,线条加粗表示被淘汰选手。}
    \label{fig:hdi_band_zh}
\end{figure}


\subsection{小结}
\begin{itemize}[itemsep=0.2em]
    \item \textbf{输出:} 可行区间、后验均值与 95\% HDI。
    \item \textbf{识别强度:} 接受率与边缘概率用于量化不确定性。
    \item \textbf{下游:} 后验样本进入反事实评估与机制设计。
\end{itemize}


% Section 5: Model 2 (Chinese)

\section{模型二:反事实赛制评估}
\label{sec:model2_zh}

\noindent\textbf{\textit{对应问题 2:}} 固定粉丝意愿与评委分数,替换规则权重,评估赛制公平性与稳定性。

\subsection{评估指标}
\begin{itemize}[itemsep=0.2em]
    \item \textbf{技术公平}:机制分数与评委分数的 Kendall 相关。
    \item \textbf{观众参与}:是否与粉丝最低票一致。
    \item \textbf{稳定性}:淘汰分布的熵稳定度。
    \item \textbf{民主赤字}:排名制相对百分制的反转概率。
\end{itemize}

\subsection{结果}
\begin{figure}[H]
    \centering
    \includegraphics[width=0.72\textwidth]{figures/fig_q2_counterfactual_matrix.pdf}
    \caption{\textbf{DAWS 基础规则在多指标下最接近平衡(高亮)。}}
    \label{fig:cf_matrix_zh}
\end{figure}

\begin{figure}[H]
    \centering
    \includegraphics[width=0.68\textwidth]{figures/fig_q2_save_sensitivity.pdf}
    \caption{\textbf{民主赤字随 Judge Save 软度变化。} 阴影带为 S28+ 参考区间。}
    \label{fig:save_sensitivity_zh}
\end{figure}

\begin{figure}[H]
    \centering
    \includegraphics[width=0.72\textwidth]{figures/fig_q2_democratic_deficit.pdf}
    \caption{\textbf{S28 规则切换后赤字走势被重新分段(阴影)。} 排名制的信息损失。}
    \label{fig:deficit_zh}
\end{figure}

% Section 6: Model 3 (Chinese)

\section{模型三:特征归因与赛制设计}
\label{sec:model3_zh}

\noindent\textbf{\textit{对应问题 3--4:}} 使用 forward-chaining 的 XGBoost+SHAP 解释选手生存因素,并设计 DAWS 动态加权赛制。

\subsection{特征归因(XGBoost + SHAP)}

\begin{figure}[H]
    \centering
    \includegraphics[width=0.78\textwidth]{figures/fig_q3_forward_chaining.pdf}
    \caption{\textbf{前向验证下预测分数依然保持较高水平。} 逐赛季 forward-chaining 评估。}
    \label{fig:forward_chaining_zh}
\end{figure}

\begin{figure}[H]
    \centering
    \includegraphics[width=0.85\textwidth]{figures/fig_shap_summary.pdf}
    \caption{\textbf{评委分统计是最主要驱动(主文仅保留 Top 15)。} 其余特征移至附录以降低视觉负担。}
    \label{fig:shap_summary_zh}
\end{figure}

\begin{figure}[H]
    \centering
    \includegraphics[width=0.7\textwidth]{figures/fig_shap_interaction.pdf}
    \caption{\textbf{年龄效应呈非线性,并受舞伴层级调节。} 该图用于结构性解释,而非提升预测性能。}
    \label{fig:shap_interaction_zh}
\end{figure}

\begin{figure}[H]
    \centering
    \includegraphics[width=0.72\textwidth]{figures/fig_shap_waterfall.pdf}
    \caption{\textbf{SHAP 瀑布图(案例)。} 单周淘汰风险的关键驱动分解。}
    \label{fig:shap_waterfall_zh}
\end{figure}

\subsection{DAWS 动态权重机制}

\begin{equation}
\alpha_t = \mathrm{clip}\Big(\alpha_0 + \gamma \frac{t}{T} - \eta U_t,\ \alpha_{\min},\alpha_{\max}\Big), \quad
|\alpha_t-\alpha_{t-1}|\le \delta,
\end{equation}
其中 $U_t$ 为不确定性代理。

\begin{figure}[H]
    \centering
    \includegraphics[width=0.7\textwidth]{figures/fig_q4_alpha_schedule.pdf}
    \caption{\textbf{DAWS 将评委权重稳定在合法区间内。} 受不确定性调节并满足平滑约束。}
    \label{fig:dynamic_schedule_zh}
\end{figure}

\subsection{帕累托前沿}

\begin{figure}[htbp]
    \centering
    \begin{minipage}[t]{0.48\textwidth}
        \centering
        \includegraphics[width=\textwidth]{figures/fig_q4_pareto_frontier.pdf}
    \end{minipage}
    \hfill
    \begin{minipage}[t]{0.48\textwidth}
        \centering
        \includegraphics[width=\textwidth]{figures/fig_ternary_tradeoff.pdf}
    \end{minipage}
    \caption{\textbf{DAWS 位于设计空间的平衡区。} 左:评委一致性 vs.\ 粉丝影响力的帕累托前沿;右:公平、参与与稳定性的三元视图。}
    \label{fig:tradeoffs_zh}
\end{figure}
\begin{figure}[H]
    \centering
    \includegraphics[width=0.75\textwidth]{figures/fig_q4_noise_robustness.pdf}
    \caption{\textbf{DAWS 在小幅扰动下保持稳定。} 权重机制对扰动的稳定性。}
    \label{fig:daws_robust_zh}
\end{figure}

\noindent\textbf{建议:} 采用 DAWS 以兼顾公平、参与与稳定性。

% Section 7: Sensitivity (Chinese)

\section{敏感性分析}
\label{sec:sensitivity_zh}

\subsection{平滑先验}
- 样本数增大可收窄 HDI,但时间成本线性增加。  
- 较小 $\sigma$ 提升平滑性,但可能掩盖突变。

\subsection{DAWS 权重}
- $\alpha_t$ 受不确定性 $U_t$ 调节,并满足平滑约束。

\subsection{特征归因稳定性}
forward-chaining + $\sigma$ 扰动下,SHAP 排名 Spearman 相关保持较高一致性。

% Section 8: Evaluation (Chinese)

\section{模型评估}
\label{sec:evaluation_zh}

\subsection{一致性检验}
\begin{itemize}[itemsep=0.1em]
    \item \textbf{后验有效性:} 后验均值位于可行区间内。
    \item \textbf{预测检验:} PPC 的 Top-3 覆盖率与 Brier 分数见 \figref{fig:ppc_metrics_zh}。
    \item \textbf{识别强度:} 接受率与边缘概率衡量不确定性。
\end{itemize}

\subsection{后验预测检验}
\begin{figure}[H]
    \centering
    \includegraphics[width=0.78\textwidth]{figures/fig_q1_ppc_metrics.pdf}
    \caption{\textbf{PPC 指标在赛季间保持稳定。} Top-3 覆盖率与 Brier 分数。}
    \label{fig:ppc_metrics_zh}
\end{figure}

\subsection{计算效率}
全流程在 3,000 个 MCMC 样本/周次条件下可在数分钟内完成,满足赛前快速审计需求。

% 结论(中文)

\section{结论}
\label{sec:conclusion_zh}

\subsection{主要结论}
\begin{enumerate}[itemsep=0.2em]
    \item \textbf{双核反演 + MaxEnt/Bayesian:} LP/MILP 提供可行区间,Hit-and-Run + 随机游走先验输出后验均值与 HDI。
    \item \textbf{Assumption--Data Tension:} $S^*$ 与可行质量标记弱识别周次。
    \item \textbf{反事实评估:} 技术公平、观众参与、稳定性与民主赤字量化赛制差异。
    \item \textbf{特征归因:} forward-chaining 的 XGBoost + SHAP(Cox 互证)揭示关键驱动因素。
    \item \textbf{机制设计:} DAWS 在公平、参与与稳定性之间取得平衡。
\end{enumerate}

\subsection{建议}
\begin{enumerate}[itemsep=0.2em]
    \item \textbf{赛前审计:} 赛前运行反演与后验分析,定位高不确定周次。
    \item \textbf{DAWS 权重:} 不确定性调节的动态权重并保持平滑。
    \item \textbf{评委拯救透明化:} 公布判断准则以降低争议。
\end{enumerate}

\subsection{局限性}
\begin{itemize}[itemsep=0.1em]
    \item 未建模外部舆情冲击。
    \item 排名制赛季仍存在弱识别问题。
    \item 反事实假设粉丝行为不随规则变化。
\end{itemize}

\subsection{未来工作}
引入外部人气先验、自适应 MCMC,并构建实时审计面板。

\vspace{0.5em}
\noindent\textbf{结论要点:} 规则一致的贝叶斯重建可提供可解释、公平的审计结果,并支持动态赛制设计。



% References
\newpage
\nocite{*}
\printbibliography[heading=bibintoc, title={参考文献}]

% Save main page count
\savemainpages

% Appendices
\clearpage
\appendix
\pagenumbering{gobble}
\pagestyle{plain}
\newpage
% AI工具使用说明 (COMAP 2026)

\section*{AI工具报告}

\addcontentsline{toc}{section}{AI工具报告}

\label{appendix:ai_zh}



\textbf{使用的工具:} Claude Opus 4.5(代码架构、算法实现)、ChatGPT-4o(语法)、GitHub Copilot(代码辅助)、DeepL(翻译)。



\textbf{使用摘要:}

\begin{table}[H]

\centering\small

\begin{tabularx}{\textwidth}{l l X}

\toprule

\textbf{工具} & \textbf{任务} & \textbf{用法} \\

\midrule

Claude & 模型设计 & LP/MILP 框架公式化 \\
Claude & 代码实现 & Python \texttt{dwts\_model} 软件包 \\

Claude & 数据分析 & Money Plot 不匹配检测 \\

ChatGPT & 文本编写 & 语法润色(技术内容由团队完成) \\

\bottomrule

\end{tabularx}

\end{table}



\textbf{验证工作:} 所有的AI输出都经过了审查和测试。关键结果(确定性错误淘汰从 40 减少到 3,S32/S33 不匹配)得到了独立验证。LP/MILP 公式均对照优化教科书进行了检查。



\textbf{纯人工贡献:} 问题定义、数据收集、结果解构、政策建议、最终参数选择 ($\alpha = 0.6$, $\epsilon = 1\%$)。



\textbf{声明:} AI工具有效加速了工作流程;所有实质性的智力贡献均反映了人类团队的判断。



\vspace{1em}

\noindent\textbf{签名:} Team \#2617892


\end{document}
